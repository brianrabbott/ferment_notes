% Options for packages loaded elsewhere
\PassOptionsToPackage{unicode}{hyperref}
\PassOptionsToPackage{hyphens}{url}
%
\documentclass[
]{book}
\title{Fermenation Notes}
\author{Brian Abbott}
\date{2022-05-31}

\usepackage{amsmath,amssymb}
\usepackage{lmodern}
\usepackage{iftex}
\ifPDFTeX
  \usepackage[T1]{fontenc}
  \usepackage[utf8]{inputenc}
  \usepackage{textcomp} % provide euro and other symbols
\else % if luatex or xetex
  \usepackage{unicode-math}
  \defaultfontfeatures{Scale=MatchLowercase}
  \defaultfontfeatures[\rmfamily]{Ligatures=TeX,Scale=1}
\fi
% Use upquote if available, for straight quotes in verbatim environments
\IfFileExists{upquote.sty}{\usepackage{upquote}}{}
\IfFileExists{microtype.sty}{% use microtype if available
  \usepackage[]{microtype}
  \UseMicrotypeSet[protrusion]{basicmath} % disable protrusion for tt fonts
}{}
\makeatletter
\@ifundefined{KOMAClassName}{% if non-KOMA class
  \IfFileExists{parskip.sty}{%
    \usepackage{parskip}
  }{% else
    \setlength{\parindent}{0pt}
    \setlength{\parskip}{6pt plus 2pt minus 1pt}}
}{% if KOMA class
  \KOMAoptions{parskip=half}}
\makeatother
\usepackage{xcolor}
\IfFileExists{xurl.sty}{\usepackage{xurl}}{} % add URL line breaks if available
\IfFileExists{bookmark.sty}{\usepackage{bookmark}}{\usepackage{hyperref}}
\hypersetup{
  pdftitle={Fermenation Notes},
  pdfauthor={Brian Abbott},
  hidelinks,
  pdfcreator={LaTeX via pandoc}}
\urlstyle{same} % disable monospaced font for URLs
\usepackage{longtable,booktabs,array}
\usepackage{calc} % for calculating minipage widths
% Correct order of tables after \paragraph or \subparagraph
\usepackage{etoolbox}
\makeatletter
\patchcmd\longtable{\par}{\if@noskipsec\mbox{}\fi\par}{}{}
\makeatother
% Allow footnotes in longtable head/foot
\IfFileExists{footnotehyper.sty}{\usepackage{footnotehyper}}{\usepackage{footnote}}
\makesavenoteenv{longtable}
\usepackage{graphicx}
\makeatletter
\def\maxwidth{\ifdim\Gin@nat@width>\linewidth\linewidth\else\Gin@nat@width\fi}
\def\maxheight{\ifdim\Gin@nat@height>\textheight\textheight\else\Gin@nat@height\fi}
\makeatother
% Scale images if necessary, so that they will not overflow the page
% margins by default, and it is still possible to overwrite the defaults
% using explicit options in \includegraphics[width, height, ...]{}
\setkeys{Gin}{width=\maxwidth,height=\maxheight,keepaspectratio}
% Set default figure placement to htbp
\makeatletter
\def\fps@figure{htbp}
\makeatother
\usepackage[normalem]{ulem}
% Avoid problems with \sout in headers with hyperref
\pdfstringdefDisableCommands{\renewcommand{\sout}{}}
\setlength{\emergencystretch}{3em} % prevent overfull lines
\providecommand{\tightlist}{%
  \setlength{\itemsep}{0pt}\setlength{\parskip}{0pt}}
\setcounter{secnumdepth}{5}
\usepackage{booktabs}
\usepackage{amsthm}
\makeatletter
\def\thm@space@setup{%
  \thm@preskip=8pt plus 2pt minus 4pt
  \thm@postskip=\thm@preskip
}
\makeatother
\ifLuaTeX
  \usepackage{selnolig}  % disable illegal ligatures
\fi
\usepackage[]{natbib}
\bibliographystyle{plainnat}

\begin{document}
\maketitle

{
\setcounter{tocdepth}{1}
\tableofcontents
}
\hypertarget{introduction}{%
\chapter{Introduction}\label{introduction}}

Just a place to archive / backup my fermentation notes while learning a bit about \texttt{R} and \href{https://bookdown.org/}{bookdown}.

\hypertarget{kombucha}{%
\chapter{Kombucha}\label{kombucha}}

\hypertarget{grapefruit}{%
\section{Grapefruit}\label{grapefruit}}

\textbf{Bulk Ferment}

\emph{Sunday, 23 May}

Scoby and starter donated from Daniella.

Method: \url{https://food52.com/blog/13548-my-adventures-in-brewing-kombucha-how-you-can-do-it-too}

\begin{itemize}
\item
  6 cups boiled, then steeped with 8 bags of black tea, 20 minutes
\item
  1 cup of sugar dissolved in hot water
\item
  6 cups cold added + 1 more when added to crock
\item
  Added Daniela's mother and the starter it came in
\item
  Covered with tea towel
\end{itemize}

\textbf{Second Ferment}

\emph{Wednesday, 2 June}

\begin{itemize}
\tightlist
\item
  Bottled
\item
  Output: 5 \textasciitilde16 oz bottles
\item
  Flavor: all 5 (2 tbsp) \textasciitilde25 ml grapefruit juice
\end{itemize}

\hypertarget{ginger-and-honey}{%
\section{Ginger and Honey}\label{ginger-and-honey}}

\textbf{Bulk Ferment}

\emph{Wednesday, 2 June}

\begin{itemize}
\tightlist
\item
  Made a new batch using same instructions as listed above; kept the mother in the crock and just added new tea to it
\end{itemize}

\textbf{Second Ferment}

\emph{Saturday, June 12}

\begin{itemize}
\tightlist
\item
  Bottled second batch
\item
  Output: 5 bottles
\item
  Flavor: ginger (\textasciitilde1in worth of matchsticks) + \textasciitilde.5 teaspoon honey
\end{itemize}

\hypertarget{grapefruit-1}{%
\section{Grapefruit}\label{grapefruit-1}}

\textbf{Bulk Ferment}

\emph{Saturday, June 12}

\begin{itemize}
\tightlist
\item
  Made another batch, same instructions
\end{itemize}

\textbf{Second Ferment}

\begin{itemize}
\tightlist
\item
  \textasciitilde30 ml grapefruit juice added to each bottle
\end{itemize}

\hypertarget{grapefruit-and-orange}{%
\section{Grapefruit and Orange}\label{grapefruit-and-orange}}

\textbf{Bulk Ferment}

\emph{Tuesday, June 22}

Made another batch, same method

\textbf{Second Ferment}

\emph{Tuesday, June 27}

\begin{itemize}
\tightlist
\item
  Flavor for second ferment was one grapefruit and one orange.
\item
  \textasciitilde25 ml in each
\item
  Did a plastic bottle for first time; had meant to leave more air.
\item
  Next time fill to base of tapered neck.
\end{itemize}

\hypertarget{watermelon-lime-and-mint}{%
\section{Watermelon, Lime, and Mint}\label{watermelon-lime-and-mint}}

\textbf{Bulk Ferment}

\emph{Sunday, June 27}

Same recipe, a little extra water.

\textbf{Second Ferment}

\emph{Tuesday, July 6}

\begin{itemize}
\tightlist
\item
  Watermelon, mint, and lime.
\item
  Juiced half a watermelon.
\item
  Muddled about a dozen mint leaves with a little sugar and splash of water.
\item
  Added juice of two limes
\item
  \textasciitilde40-50 ml of this juice added to each bottle, filled with kombucha
\item
  Used old kirkland kombucha bottles from a friend and one plastic coke bottle as a carbonation gauge.
\end{itemize}

\hypertarget{peach-and-ginger}{%
\section{Peach and Ginger}\label{peach-and-ginger}}

\textbf{Bulk Ferment}

\emph{Tuesday, July 6}

Brewed new batch as usual. Added one 16 oz bottle of water to fill the crock.

Plastic bottle was hard after a few days, put in fridge. Could have let it get harder, as bottles didn't have much carbonation, but gained some in fridge (we were gone July 11-16).

\textbf{Second Ferment}

\emph{Sunday, July 18}

\begin{itemize}
\tightlist
\item
  12 day first ferment
\item
  Peaches and ginger
\item
  One plastic bottle as carbonation gauge
\end{itemize}

\hypertarget{peach}{%
\section{Peach}\label{peach}}

\textbf{Bulk Ferment}

\emph{Sunday, July 18}

No notes.

\textbf{Second Ferment}

Flavored with just peach. Opinion: ginger and peach is better.

\hypertarget{strawberry-kiwi}{%
\section{Strawberry Kiwi}\label{strawberry-kiwi}}

\textbf{Bulk Ferment}

\emph{No dates / notes}

\textbf{Second Ferment}

\emph{Sunday August 29}

Flavor: \textbf{strawberry kiwi}, Cara and Arlo liked best yet.

\hypertarget{nectarine-and-blueberry}{%
\section{Nectarine and Blueberry}\label{nectarine-and-blueberry}}

\textbf{Bulk Ferment}

\emph{Sunday August 29}

No notes.

\textbf{Second Ferment}

\emph{Saturday September 18}

Nectarine and frozen blackberries

\hypertarget{ginger-and-mint}{%
\section{Ginger and Mint}\label{ginger-and-mint}}

\textbf{Bulk Ferment}

\emph{Saturday September 18}

\textbf{Second Ferment}

Ginger and mint

\hypertarget{apple-and-ginger}{%
\section{Apple and Ginger?}\label{apple-and-ginger}}

\emph{No date / notes}

\textbf{Second Ferment}

Apple and ginger maybe?

\hypertarget{mystery}{%
\section{Mystery?}\label{mystery}}

\textbf{Bulk Ferment}

\emph{Sunday October 31}

\textbf{Second Ferment}

\emph{No date}

Ginger and lemon

\hypertarget{pear-and-ginger}{%
\section{Pear and Ginger}\label{pear-and-ginger}}

\textbf{Bulk Ferment}

\emph{No date / notes}

Left for a few months. Checked periodically to ensure nothing looked off.

\textbf{Second Ferment}

\emph{Sunday January 23, 2022}

Had reduced to \textless4 bottles during F1, trying f2 for science. One plastic bottle for carbonation guage.

\begin{itemize}
\tightlist
\item
  Flavor: pear and ginger

  \begin{itemize}
  \tightlist
  \item
    1 cc size cubes
  \item
    Sugar: 1/2 tsp in each bottle
  \end{itemize}
\end{itemize}

\emph{Unknown date}

Update: none carbonated, f2 for a couple of weeks at room temp. Tasted fine though, maybe a bit more tart than usual.

\hypertarget{strawberry}{%
\section{Strawberry}\label{strawberry}}

\textbf{Bulk Ferment}

\emph{Sunday January 23, 2022}

\textbf{Second Ferment}

\emph{Tuesday, May 17}

Again let sit for several months.

\begin{itemize}
\tightlist
\item
  Quantity: 4 soda bottles filled to top of label
\item
  Flavor: about 12 strawberries and a sizable scoop full of sugar

  \begin{itemize}
  \tightlist
  \item
    Macerated the strawberries in the sugar. \textasciitilde60 ml per bottle
  \end{itemize}
\end{itemize}

\emph{Thursday may 26}

All four soda bottles firm enough to not give, refrigerated

\begin{itemize}
\tightlist
\item
  Thoughts: a bit too sweet
\end{itemize}

\hypertarget{ginger-beer}{%
\chapter{Ginger Beer}\label{ginger-beer}}

\hypertarget{batch-1}{%
\section{Batch 1}\label{batch-1}}

\textbf{Bug}

\emph{Early September 2021}

Started a ginger bug, chlorine evaporated water from starter sourdough jug. Arbitrary amount of grated ginger in small amount of initial water. After about a week added more ginger and a bit of sugar. Started shaking it, got bubbly.

\textbf{Bulk Ferment}

\emph{September 18 2021}

\begin{itemize}
\item
  Cut up a few chunks of skinned ginger and simmered in 2 qts of tap water for \textasciitilde15 minutes.
\item
  Strained out ginger solids
\item
  Added 2 cups sugar, stirred too dissolve
\item
  Added 2 qts of tap and let cool
\item
  Added strained ginger bug
\item
  F1 in cloth covered crock
\end{itemize}

\textbf{Second Ferment}

No notes, but it was pretty good. Definitely different than commercial versions.

\hypertarget{batch-2}{%
\section{Batch 2}\label{batch-2}}

\textbf{Bug}

\emph{May 24-ish, 2022}

Started bug with occasional feedings of arbitrary amounts. Grew vigorous after a few days, but ran out of ginger mid-covid quarantine so had to wait for a grocery pick up and bug grew stagnant. Revived with a couple (two) feedings, noticed a \emph{few} bubbles on May 31st. All feedings used de-chlorinated water left out overnight and ginger (skin included) grated on the fine side of a box grater.

\textbf{Bulk Ferment}

\emph{May 31, 2022}

Loosely followed \href{https://www.wildfermentation.com/the-art-of-fermentation/}{The Art of Fermentation} method.

\begin{itemize}
\item
  Simmered 2 quarts of water with \textasciitilde5 in of coursely grated ginger (skin included, not sure if appropriate to have left it on) for about 15 minutes
\item
  Dissolved \sout{2 cups} 400 g sugar
\item
  Added 2 quarts cool water to dilute, tasted and let cool to room temp for several hours (very warm day)
\item
  Added bug (slightly less than \textasciitilde2 cups)
\item
  Mixed then strained through fine mesh sieve
\end{itemize}

\textbf{Output}

Split perfectly between a half-gallon mason jar and a 2 liter kliner jar.

\end{document}
